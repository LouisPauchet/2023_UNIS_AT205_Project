\begin{titlepage}
\vbox{}
\vbox{}
\begin{center}
% Upper part of the page
\includegraphics[width=0.40\textwidth]{Images/UNIS_logo.png}\\[2cm]
\textsc{\LARGE Department of Arctic Technology}\\[1.5cm]
\textsc{\Large AT-205 -- Frozen Ground Engineering for Arctic Infrastructures}\\[0.5cm]
\vbox{}

% Title
\HRule \\[0.5cm]
{ \huge \bfseries Investigating the underground using Ground Penetrating Radar around the 2018 landslide area }\\[0.5cm]
\HRule \\[1.5cm]

% Author
\large
\emph{Author: Louis PAUCHET (107893)} 

\emph{Supervisor: Alexey Shestov} 

\vspace{2.5cm}

{\large May, 2023}

\vfill

% Bottom of the page

\end{center}

\doclicenseThis

\end{titlepage}

\newpage

\vspace*{\fill}
\begin{center}
    \huge \emph{Thank you so much to}
\end{center}

\emph{Alexey Shestov} for his help in settings up the GPR and his advice all along this project as well as the \emph{Artic Technology Department} to provide us this wonderful study environment;

\emph{Knut Ivar Lindland Tveit} for the organisation of the field activity and his support during the field work;

\emph{Stig Magnus Lunde} for his help to find the required equipment to carry the GPR on the survey area as well as the whole section for \emph{Operation and Field Safety} for the support during this field work and all the others;

\emph{Georgios Tassis} from the Norges geologiske undersøkelse (NGU) for teaching us the basis on how to use a ground penetrating radar;

\emph{Gunnhild Næss} for taking picture in the town during the survey test.
\vspace*{\fill}

\newpage