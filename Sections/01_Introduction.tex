\section{Introduction}

Longyearbyen, serving as the primary gateway to the Svalbard archipelago, has experienced significant inhabitant growth over the years. From 2085 residents in 2009, the population has risen to 2530 in 2023 (according to ssb.no). However, the population figures mentioned do not account for the transient influx of people during the summer season due to numerous cruise ships arriving regularly. Consequently, concerns about the health and safety of the residents have emerged in the past decade. This is especially true in light of various natural events, such as avalanches and landslides, which are now being closely monitored to prevent any repetition of the 2015 disaster that killed two people and caused significant property damages. The focus is now on predicting such events to facilitate timely evacuations.

Given the current scenario of rapid environmental transformation caused by global warming, characterized by higher temperatures and significant alterations in precipitation patterns, the understanding of events such like land slide is very important and we need to to determine the underground geological and stratigraphic structure.

In this project, we will use a Ground Penetrating Radar in order to investigate the underground structure on the place where a landslide happened in 2018 \cite{Landslide2018Longyeabyen} with the main objective to provide open data to understand the subsurface and help to develop a model like in the PermaMeteoCommunity project \cite{PermaMeteoCommunity}.

Finally, the second objective is to gain some experience on the ground penetrating radar and try to identify some characteristics structure in the subsurface.

