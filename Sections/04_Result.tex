\newpage
\section{Results and interpretation}

We use the same code for the interpretation of all lines and the meaning of the different colors are presented bellow on the figure \ref{fig:legend}.

\begin{figure}
    \centering
    \includegraphics{}
    \caption{Legend for the interpretation of the figure }
    \label{fig:legend}
\end{figure}

\subsection{Test survey}

\subsection{Main Survey}

\paragraph{Results from the lines parallel to the slope} First we will present the lines realised parallel to the slope because they permit to understand some of the limitation of the GPR and of the ground.

\begin{figure}
    \centering
    \includegraphics[width=\linewidth]{Images/00_Results/line1_edited.jpg}
    \caption{Line 1 with interpretations, legend on the figure \ref{fig:legend}}
    \label{fig:line1}
\end{figure}

\begin{figure}
    \centering
    \includegraphics[width=\linewidth]{Images/00_Results/line16_edited.jpg}
    \caption{Line 16 with interpretations, legend on the figure \ref{fig:legend}}
    \label{fig:line16}
\end{figure}

\begin{figure}
    \centering
    \includegraphics[width=\linewidth]{Images/00_Results/line17_edited.jpg}
    \caption{Line 17 with interpretations, legend on the figure \ref{fig:legend}}
    \label{fig:line17}
\end{figure}

\begin{figure}
    \centering
    \includegraphics[width=\linewidth]{Images/00_Results/line18_edited.jpg}
    \caption{Line 18 with interpretations, legend on the figure \ref{fig:legend}}
    \label{fig:line18}
\end{figure}

\begin{figure}
    \centering
    \includegraphics[width=\linewidth]{Images/00_Results/line19_edited.jpg}
    \caption{Line 19 with interpretations, legend on the figure \ref{fig:legend}}
    \label{fig:line19}
\end{figure}

\begin{figure}
    \centering
    \includegraphics[width=\linewidth]{Images/00_Results/line21_edited.jpg}
    \caption{Line 21 with interpretations, legend on the figure \ref{fig:legend}}
    \label{fig:line21}
\end{figure}

