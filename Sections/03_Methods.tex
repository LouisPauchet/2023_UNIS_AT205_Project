\newpage
\section{Methodology} \label{Section:Methodology}

%In this part, we will describe everything needed to realise the survey and reproduce the same data. Let's start with the GPR setup we use.

\subsection{GPR Setup}

The setup to run a GPR in a deep slope as describe in \ref{Paragraph:SiteDescription} should easy to carry and quite compact but compelling with all the requirement of the GPR. The picture on the figure \ref{fig:PictureGunnhild} show all the element as listed bellow.

\begin{itemize}
    \item \emph{The pulk} will be the support for the GPR receiver and transmitter. The challenge is to fix both antenna on it and the simpler solution was to tape the antenna on the pulk.
    \item \emph{The antennas} placed directly above the pulk (rectangular black plate)
    \item \emph{The transmitter}, black box above the back antenna on the figure, \ref{fig:PictureGunnhild} will generate the signal send by a fiber optique from the DVL\footnote{Data logger}. The transmitter we used require to disconnect the output wire.
    \item \emph{The receiver}, yellow box above the front antenna, will be the most critical part of the set up. It's the receiver which will receive the reflection from the ground be it will also be sensible to all external perturbation. That's why for the 'real' survey, we choose to invert the receiver and the transmitter to keep the receiver the farthest away from all the rest of electronics devices.
    \item \emph{The GPS receiver} located on the top of the backpack will permit to have the position of each measurement and to correct the position of the lines during the processing. It will also give the indication about the elevation which is very important in a slope survey.
    \item \emph{The DVL} will received and send all the data and save it. It also permit to set up the whole GPR (Frequency, Antenna separation... as presented in \ref{SubSection:settings})
\end{itemize}

\begin{figure} [H]
    \centering
    \includegraphics[width=0.7\linewidth]{Images/00_Methodology/PictureGunnhild.jpg}
    \caption{GRP setup used for the survey during the tests (see \ref{Subsection:TestSurvey}) realised in the main street of Longyearbyen. \emph{Photo : Gunnhild Næss - UNIS}}
    \label{fig:PictureGunnhild}
\end{figure}

\subsection{Test survey} \label{Subsection:TestSurvey}

In order to try the all setup which was new (especially the data logger), we went in the streets of Longyearbyen to do some lines just for tests. This survey permit to fix some issues with the GPS and think about the a better organisation of the receiver and transmitter for the "real survey". It's during these test that Gunnhild Næss took the picture presented on the figure \ref{fig:PictureGunnhild}. We only present one of the line realised during this survey because it is very characteristics of a buried pipe (see figure \ref{fig:testLine} in the \ref{section:result}).

\begin{wrapfigure}{r}{5.5cm}
    \centering
    \includegraphics[width=5.5cm]{Images/00_Methodology/LandSlide2018.jpg}
    \caption{Landslide in Longyearbyen in 2018 \emph{Photo: Alexander Lembke} \cite{Landslide2018Longyeabyen}}
    \label{fig:Landslide2018}
\end{wrapfigure}

\subsection{Field work organisation}


\paragraph{Site description} \label{Paragraph:SiteDescription}

One of the best concern in Longyearbyen 




\begin{figure}
    \centering
    \includegraphics[width=\linewidth]{Images/00_Methodology/GeographicSituation.jpg}
    \caption{Localisation of the survey area in Longyearbyen}
    \label{fig:Location}
\end{figure}


\paragraph{Geology} Using a GPR as for goal to "see" the subsurface and so some geological considaration are required. We had the information from the survey on the Geological map of Svalbard \cite{Atakan2015GeoscienceSvalbard} and the result is presented on the figure \ref{fig:GeologicalBackground}. This map give us the information that the bedrock will be most likely sandstone and shale which give us an information about the expected velocity about 0.130 m/ns \cite{GPRAnalysis}.

\begin{figure}
    \centering
    \includegraphics[width=\linewidth]{Images/00_Methodology/GeologicalSituationMap.jpg}
    \caption{Geological background under the survey area \cite{Atakan2015GeoscienceSvalbard}}
    \label{fig:GeologicalBackground}
\end{figure}

\paragraph{Weather conditions} The field work take place in a very perturbed but sunny spring period. The day of the field work, April 28th, the weather was very sunny and warm with a temperature reaching 0.5°C. The figure \ref{fig:weather} show the history of the weather 2 weeks around the field work. We see an high variability of the temperature (between -18 and 2°C) as well as a high variability of wind.

\begin{figure}
    \centering
    \includegraphics[width=\linewidth]{Images/00_Methodology/WeatherConditions.png}
    \caption{Weather condition around Longyeabyen from the UNIS weather station in Adventdalen between April 24 and May 7 2023}
    \label{fig:weather}
\end{figure}

\paragraph{Safety consideration} In the arctic, safety for field operation is one of the most important concern, even when the field activity are close to the town. According to cold weather, the risk of terrain slide as it append in 2018 is very unlikely, because the active layer is still frozen. But a more important concern is avalanches because of the deep slope of the survey and even with a risk of 2 over 5. That's why we have with us all the avalanche safety equipment (avalanche beacon, probe and shovel).
Finally, we stay close for a very frequented area, in town and we always stay under phone coverage in case of any emergency.

\paragraph{Planned grid}

The initial plan was to realise a grid as presented in blue on the figure \ref{fig:GPRLines} in order to have the better resolution possible under the ground. The issue was it was very hard to follow the line with the GPS and the irregularity of the surface increase this complexity. Finally, and it was the may concern, the initial grid was very ambitious about the physical activity, the whole distance being walk. 

\begin{figure}
    \centering
    \includegraphics[width=\linewidth]{Images/00_Methodology/GPR LIne.jpg}
    \caption{Lines planned and realised during the main survey}
    \label{fig:GPRLines}
\end{figure}

\subsection{Processing}