\newpage
\section{Ground penetrating radar}

\subsection{Principle}

Ground penetrating radar, or GPR, is a technology to investigate the subsurface using radars. A transmitter (in orange marked Tx on the figure \ref{fig:GPR}) generates an electromagnetic wave which is transmitted in the ground via an antenna. This signal will propagate in to the ground at different velocities depending on the material ($0.13 m/ns$ for sandstone or $0.5 m/ns$ for fresh water \cite{GPRAnalysis}) and will be both reflected and transmitted on the interface between two different materials. The reflected part will be received by the reception antenna connected to the receiver (Blue marked Rx on the figure \ref{fig:GPR}).

\begin{figure}[h]
    \centering
    \includegraphics[width=0.8\linewidth]{Images/00_Theory/GPR Diagramm.drawio.png}
    \caption{Ground penetrating radar principle}
    \label{fig:GPR}
\end{figure}

Now we have the basic of the principle of GPR, we could consider different positioning for both antenna. Indeed, we could choose to keep the receiver fixed and the transceiver moving, at the opposite, both transmitter and receiver moving around a common mid point... In this report, we will only use the set up with the transmitter and the receiver moving together, antenna perpendicular to the moving direction.
This usage of GPR permits to investigate a big area by doing multiple lines. On each line, each measurement point will represent the ground under the GPR and so we can have an overview of the subsurface along the line. \cite{Rnning2023GroundPerformance}

\subsection{Limitations}

As the GPR uses a range of frequency from Mhz to Ghz which is also the range for many communication and electronics devices, GPR is very sensitive to all radio-electric perturbations. So it must be operated far from any noise source as electric grid, GSM antenna...

An other limitation is that propagation characteristics depend on the material where the electromagnetic wave propagates. Consequently the energy lost by the wave during it propagation depend on the material as well as the velocity. So for the same time window\footnote{Time during the receiver is waiting the wave to come back before emitting a new signal. See \ref{SubSection:settings}}, if the signal propagates fast or slow, the depth where we can see will be change. Also, a material with a huge attenuation limits the depth of the survey because the signal will lose very quickly the power.

Finally, every saline environments are almost impossible to investigate with GPR because the conductivity of salty water is very good, too good\cite{UnderstandingDetection}.

\subsection{Settings} \label{SubSection:settings}


\paragraph{Tension} The tension gives an indication about the energy given to the emitted signal. Higher the tension is, deeper the signal will propagate but also higher the GPR will disturb the communication devices. The maximum tension allowed for civil survey is 400V even if in the past some GPR with a tension of 1000V were used. 

\paragraph{Frequency} The frequency is one of the most important parameter for a GPR survey because it defines the resolution of the GPR. The figure \ref{fig:GPRFrequency} show that more precise the survey will be, less deep the penetration will be and higher the frequency will be needed. 

\begin{figure}[h]
    \centering
    \includegraphics[width=0.6\textwidth]{Images/00_Theory/SS_FrequencyDepthResolution.jpg}
    \caption{Exemple of resolution length and depth versus frequency ($v=0.1m/ns$) from Sensor and Software \cite{UnderstandingDetection}}
    \label{fig:GPRFrequency}
\end{figure}

\textcolor{red}{Relecture}

\paragraph{Number of stacks} The stacking consist to measure the same data many times to take the mediam or mean value and reduce the noise. The number of stacks should be the higher possible but will depend of the capabilities of the data-logger and of the different others parameters

\paragraph{Time windows} As defined before, the time windows is the time during which the receiver will wait the signal reflected in the ground. If we use a small time windows, we risk to missed some information. At the opposite, if the time windows is too large, the data logger will record only noise which will reduce the capability of the data logger.

\paragraph{Sampling interval} The reception antenna received an analog signal coming from the ground but the data-logger only record digital signal. It is so important to have an sampling interval which correspond to signal. A sampling interval too low will conduct to having a completely different signal. The important criteria is the Nyquist Criteria which indicate that the sampling interval should be smaller than the maximum expected period in order to catch all the frequency. 

\begin{figure}[H]
    \centering
    \includegraphics[width=0.8\linewidth]{Images/00_Theory/SampleRate.png}
    \caption{Sample rate illustration}
    \label{fig:SampleRate}
\end{figure}

On the example from the figure \ref{fig:SampleRate}, we see that the sample interval of 3 TU\footnote{TU = Time Unit} is to low where the sample interval of 1 TU is better.

\paragraph{Antenna separation} The antenna separation will influence the quality of the signal due to the couplage between the two antenna. In order to set this parameter, we will just refer to the manufacturer documentation.

\paragraph{Antenna orientation} The orientation of both antenna will be choose for practical purposes : both antenna perpendicular to the walking direction even if all the combination are possible, just complicated to set up on a pulk (See \ref{Section:Methodology}). We choose to have the reception antenna the farthest away from any electronic devices.


\paragraph{Sampling distance} The sampling distance will be the distance between two trace, this will give the horizontal resolution. Lower this distance is higher the horizontal resolution will be but slower will be the speed of the displacment.

\paragraph{Velocity} The velocity is a secondary parameter for a GPR because in many cases the velocity is determined using GPR data. The important thing to know is that the velocity will convert the time windows into a depth windows and so determine the depth where the GPR is able to see.


