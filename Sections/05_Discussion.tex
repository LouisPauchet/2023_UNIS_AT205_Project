\newpage
\section{Discussions}

The GPR investigation confirmed the nature of the upper sedimentary subsurface, characterized by the presence of multiple layers due to many avalanche and landslide deposits through the geological time. This layering is crucial to understand the history of the ground because each layer correspond to a geological event.

Unfortunately, the ground penetrating radar is very insufficient to be able to precisely characterise each echo even if we are able to do some hypotheses. To increase the accuracy of the characterisation of the subsurface, additional investigations will be required by people specialized in geological survey and stratigraphy.

The survey confirms also one of the limitation of the GPR in term of perturbation with the "nice view" of the public light electric infrastructure close to the road on the figures \ref{fig:line1} and \ref{fig:line21} representing the lines 1 and 21. Even if the perturbation due to the electric wire is the most easy and likely hypotheses, we should confirm it using the mapping of the road infrastructure around the Vei 600.

More globally, the limitation of the GPR in that location is the time and man power to realise the survey. Indeed, the survey as planned and the performed lines show on the figure \ref{fig:GPRLines} is very characteristics of the difficulties to operate a GPR in a steep slope with a lot of rocks. Especially, using a pulk to carry the transmitter and receiver makes almost impossible to operate the setup above rocky area. A the GPR mounted on a wheel rig should have been very complicated to operate on the snow. An other way to move the antennae manually between each measurement should have led to an more accurate survey (longer time for each measurement) but increase a lot the time needed.

The resolution of the data is quite good even if there is still a lot of noise due mainly to the little amount of stacks (64) done because the receiver is now quite old.