\newpage
\section{Discussions}

The GPR investigation confirmed the nature of the upper sedimentary subsurface, characterized by the presence of multiple layers consequence of many avalanche and landslide deposit through the geological time. This layering is very important to understand the history of the ground because each layer correspond two an geological event.

Unfortunately, the ground penetrating radar is very insufficient to be able to precisely characterise each echo even if we are able to present some hypotheses. To be able to increase the precision of the characterisation of the subsurface, more investigation will be necessary by people specialized in geological survey and stratigraphy.

The survey confirm also one of the limitation of the GPR in term of perturbation with the nice view of the electric grid close to road on the figures \ref{fig:line1} and \ref{fig:line21} representing the lines 1 and 21. Even if the perturbation due to the electric wire is the most easy and likely hypotheses, we should try to confirm it using the mapping of the road infrastructure around the Vei 600.

More globally, the limitation of the GPR in that location is the time and man power to realise the survey. Indeed, the survey as it was plan and the final lines show on the figure \ref{fig:GPRLines} is very characteristics of the difficulty to operate a GPR in a deep slope with a lot of rocks. Especially, using a puck to carry the transmitter and receiver make almost impossible to operate the setup above rocky area and also the GPR mounted on a wheel rig is very complicated to operate on the snow. The last way to move the antenna manually between each measurement permit an more accurate survey (longer time for each measurement) but increase a lot the time needed for the survey.

The resolution of the data is quite good even if there is still a lot of noise due mainly by the little among of stacks realised because the receiver is now quite old.